\documentclass[index.tex]{subfiles}

\begin{document}
\newcommand{\CMName}{\textsf{Coercion}}
\newcommand{\CMV}{\ensuremath{c}}

\newcommand{\CId}{\ensuremath{\textsf{id}}}
\newcommand{\CFail}{\ensuremath{\textsf{fail}}}
\newcommand{\CEmb}[1]{\ensuremath{#1!}}
\newcommand{\CProj}[1]{\ensuremath{#1?}}
\newcommand{\CFun}[2]{\ensuremath{\textsf{fun} ~#1 ~#2}}
\newcommand{\CSeq}[2]{\ensuremath{#1; #2}}

\newcommand{\DTMName}{\textsf{Dynamic Tag}}
\newcommand{\DTMV}{\ensuremath{g}}
\newcommand{\DTBase}{\ensuremath{b}}
\newcommand{\DTFun}{\ensuremath{\textsf{fun}}}

\newcommand{\IEMName}{\textsf{Internal Expression}}
\newcommand{\IEMV}{\ensuremath{d}}
\newcommand{\IECast}[2]{\ensuremath{#1 \langle #2 \rangle}}

\subsection{Coercion calculus for holes}
\label{sec:coercion}
This section contains experiments with a coercion calculus for \Hazelnut{}. Specifically, we give an
augmented internal language where casts are given by coercions and a conversion to it from the original
internal language. Refer to the \HazelnutLive{} paper \cite{omar2019} for definitions of types and
expressions, which we do not reproduce here.

\Cref{fig:coercion-syntax} gives the syntax of coercions and shows how they fit into expressions,
eliding most of the expression forms.

\begin{figure}[htb!]
  \[\begin{array}{rrcl}
    \CMName  & \CMV  & \Coloneqq & \CId \mid \CFail
                                 \mid \CEmb{\DTMV} \mid \CProj{\DTMV} 
                                 \mid \CFun{\CMV}{\CMV} \mid \CSeq{\CMV}{\CMV} \\
    \DTMName & \DTMV & \Coloneqq & \DTBase \mid \DTFun \\
    \IEMName & \IEMV & \Coloneqq & \cdots \mid \IECast{\IEMV}{\CMV}
  \end{array}\]
  %
  \caption{Coercion language}
  \label{fig:coercion-syntax}
\end{figure}

\begin{figure}[htb!]
  \begin{mathpar}
  \end{mathpar}
  %
  \caption{Coercion type assignment}
  \label{fig:coercion-types}
\end{figure}
 
\end{document}
