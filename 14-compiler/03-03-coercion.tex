\documentclass[index.tex]{subfiles}

\begin{document}
\newcommand{\coerce}{\ensuremath{\textsf{coerce}}}
\newcommand{\coerces}[2]{#1 \rightsquigarrow #2}
\newcommand{\ctxCoercesType}[4]{\ensuremath{\ctxAssignType{#1}{#2}{\coerces{#3}{#4}}}}

\newcommand{\TMName}{\textsf{Type}}
\newcommand{\TMV}{\ensuremath{\tau}}
\newcommand{\TBase}{\ensuremath{b}}
\newcommand{\THole}{\ensuremath{\SyEHole{}{}}}
\newcommand{\TArrow}[2]{\ensuremath{#1 \to #2}}

\newcommand{\CMName}{\textsf{Coercion}}
\newcommand{\CMV}{\ensuremath{c}}

\newcommand{\CId}{\ensuremath{\textsf{id}}}
\newcommand{\CFail}{\ensuremath{\textsf{fail}}}
\newcommand{\CEmb}[1]{\ensuremath{#1!}}
\newcommand{\CProj}[1]{\ensuremath{#1?}}
\newcommand{\CFun}[2]{\ensuremath{\textsf{fun} ~#1 ~#2}}
\newcommand{\CCompose}[2]{\ensuremath{#1; #2}}

\newcommand{\DTMName}{\textsf{Dynamic Tag}}
\newcommand{\DTMV}{\ensuremath{g}}
\newcommand{\DTBase}{\ensuremath{\TBase}}
\newcommand{\DTFun}{\ensuremath{\textsf{fun}}}

\newcommand{\IEMName}{\textsf{Internal Expression}}
\newcommand{\IEMV}{\ensuremath{d}}
\newcommand{\IECast}[2]{\ensuremath{#1 \langle #2 \rangle}}

\subsection{Coercion calculus for holes}
\label{sec:coercion}
This section contains experiments with a coercion calculus \cite{herman2010} for \Hazelnut{}.
Specifically, we give an augmented internal language where casts are given by coercions and a
conversion to it from the original internal language. Refer to the \HazelnutLive{} paper
\cite{omar2019} for definitions of types and expressions, which we do not reproduce here.

Following \textcite{herman2010}, we define the partial function $\coerce : \TMName \times \TMName
\to \CMName$ which gives coercions between two consistent types:
%
\newcommand{\coerceRowIs}[3]{\ensuremath{\coerce(#1, #2) &=& #3}}
\[\begin{array}{rcl}
  \coerceRowIs{\TMV}{\TMV}{\CId} \\
  \coerceRowIs{\TBase}{\THole}{\CEmb{\DTBase}} \\
  \coerceRowIs{\THole}{\TBase}{\CProj{\DTBase}} \\
  \coerceRowIs{\TArrow{\TMV_1}{\TMV_2}}{\TArrow{\TMV_1'}{\TMV_2'}}{\CFun{\coerce(\TMV_1', \TMV_1)}{\coerce(\TMV_2, \TMV_2')}} \\
\end{array}\]

\Cref{fig:coercion-syntax} gives the syntax of coercions and shows how they fit into expressions,
eliding most of the expression forms, and \Cref{fig:coercion-types} gives rules for type
assignment. Note that the forms and typing, as well as most other constructs and lemmas, are
borrowed from \textcite{herman2010}. However, there are a few key differences:
%
\begin{itemize}
  \item In agreement with \HazelnutLive's semantics, $\CFail$ coercions (which replace the explicit
    failed cast form) must coerce two \emph{inconsistent ground} types.
\end{itemize}

\begin{figure}[htb!]
  \[\begin{array}{rrcl}
    \CMName  & \CMV  & \Coloneqq & \CId \mid \CFail
                                 \mid \CEmb{\DTMV} \mid \CProj{\DTMV} 
                                 \mid \CFun{\CMV}{\CMV} \mid \CCompose{\CMV}{\CMV} \\
    \DTMName & \DTMV & \Coloneqq & \DTBase \mid \DTFun \\
    \IEMName & \IEMV & \Coloneqq & \cdots \mid \IECast{\IEMV}{\CMV}
  \end{array}\]
  %
  \caption{Coercion language}
  \label{fig:coercion-syntax}
\end{figure}

\begin{figure}[htb!]
  \begin{mathpar}
    \judgment{ }{
      \ctxCoercesType{}{\CId}{\TMV}{\TMV}
    }{CId}

    \judgment{
      \isGround{\TMV} \\
      \isGround{\TMV'} \\
      \isNotConsistent{\TMV}{\TMV'}
    }{
      \ctxCoercesType{}{\CFail}{\TMV}{\TMV'}
    }{CFail}

    \judgment{ }{
      \ctxCoercesType{}{\CEmb{\DTBase}}{\TBase}{\THole}
    }{CB!}

    \judgment{ }{
      \ctxCoercesType{}{\CProj{\DTBase}}{\THole}{\TBase}
    }{CB?}

    \judgment{ }{
      \ctxCoercesType{}{\CEmb{\DTFun}}{(\TArrow{\THole}{\THole})}{\THole}
    }{CFun!}

    \judgment{ }{
      \ctxCoercesType{}{\CProj{\DTFun}}{\THole}{(\TArrow{\THole}{\THole})}
    }{CFun?}

    \judgment{
      \ctxCoercesType{}{\CMV_1}{\TMV_1}{\TMV_1'} \\
      \ctxCoercesType{}{\CMV_2}{\TMV_2}{\TMV_2'}
    }{
      \ctxCoercesType{}{\CFun{\CMV_1}{\CMV_2}}{\TArrow{\TMV_1}{\TMV_2}}{\TArrow{\TMV_1'}{\TMV_2'}}
    }{CFun}

    \judgment{
      \ctxCoercesType{}{\CMV_1}{\TMV}{\TMV_1} \\
      \ctxCoercesType{}{\CMV_2}{\TMV_1}{\TMV_2}
    }{
      \ctxCoercesType{}{\CCompose{\CMV_1}{\CMV_2}}{\TMV}{\TMV_2}
    }{CCompose}

    \judgment{
      \ctxCoercesType{}{\CMV}{\TMV}{\TMV'} \\
      \ctxAssignType{\ctx}{\IEMV}{\TMV}
    }{
      \ctxAssignType{\ctx}{\IECast{\IEMV}{\CMV}}{\TMV'}
    }{TACast}
  \end{mathpar}
  %
  \caption{Coercion type assignment}
  \label{fig:coercion-types}
\end{figure}

\begin{lemmat}[name=Well-typed coercions] \
  \begin{enumerate}
    \item $\isConsistent{\TMV}{\TMV'}$ if and only if $\coerce(\TMV, \TMV')$ is defined.
    \item If $\CMV = \coerce(\TMV, \TMV')$, then $\ctxCoercesType{}{\CMV}{\TMV}{\TMV'}$.
  \end{enumerate}
\end{lemmat}

The original [TACast] rule is replaced by a new one for coercion-based casts.
 
\end{document}
